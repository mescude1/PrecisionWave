\documentclass{article}
\usepackage{minted}

\title{EAFIT University - Department of Information and Computer Sciences}
\author{Mauricio Escudero Restrepo \\
        César Esteban Peñuela Cubides \\
        Diego Mesa Ospina}
\date{August 2024}

\begin{document}

\maketitle

\section{Objetivo}
Review the progress of the final project and the work methodologies of the workgroups

\section{Course}
Numerical Analisis

\section{Responsible Faculty Member}
Edwar Samir Posada Murillo

\section{Current Report Delivery Date}
October 1st 2024
    
\section[]{Numerical Methods}
    The following numerical methods will be presented in this document in the following manner, first a pseudocode 
    version of the method algorith will be presented, after the code for the implementation of the method in both 
    languages selected will be presented, finally proof for the execution and results of the methods will be provided.
    
    \subsection{Incremental Search}

    Incremental Search Method is a numerical method used to find the roots of a function
    (i.e., where the function equals zero).
    This method works by iteratively narrowing down an interval where a root
    lies.
    The goal is to find an approximation of the root with increasing precision.

    \subsection{Method Implementation}
    \subsection{Python}
    \begin{minted}{python}
        def incremental_search(f, x0, h, Nmax):
    """
    This program finds an interval where f(x) has a sign change using the incremental search method.
    Inputs:
    f: continuous function
    x0: initial point
    h: step
    Nmax: maximum number of iterations

    Outputs:
    a: left endpoint of the interval
    b: right endpoint of the interval
    iter: number of iterations
    """

    # Initialization

    xant = x0
    fant = f(xant)
    xact = xant + h
    fact = f(xact)
    result_array = []
    # Loop
    for i in range(1, Nmax+1):
        if fant * fact < 0:
            result = {
                'i': i,
                'x_i': xact,
                'f_xi': fact,
                'e': abs(xact - xant)
            }
            result_array.append(result)
            break
        result = {
            'i': i,
            'x_i': xact,
            'f_xi': fact,
            'e': abs(xact - xant)
        }
        result_array.append(result)
        xant = xact
        fant = fact
        xact = xant + h
        fact = f(xact)

    # Result delivery

    a = xant
    b = xact
    iter = i
    result_data_frame = pd.DataFrame(result_array)
    return a, b, iter, result_data_frame
        \end{minted}


\end{document}
